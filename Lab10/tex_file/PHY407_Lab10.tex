\documentclass{article}
\usepackage[utf8]{inputenc}
\usepackage{graphicx}
\usepackage{geometry}
\usepackage{amsmath}
\usepackage{amsfonts}
\usepackage{float}
\usepackage{caption}
\usepackage{subcaption}
\usepackage{enumitem}

\geometry{left=25mm, top=25mm, right=25mm, bottom=25mm}

\title{PHY407 Lab 10}
\author{Pierino Zindel (1002429703) and Hayden Johnson (1002103537)}
\date{November 23, 2018}

\begin{document}

\maketitle

\noindent \textbf{Distribution of work:} Question 1 was completed by Pierino. Questions 2 and 3 were completed by Hayden.

\section{Brownian Motion and Diffusion Limited Aggregation}

\section{Volume of a 10-dimensional Hypersphere}

We seek to compute the volume of a 10-dimensional hypersphere, using the Monte Carlo mean value integration method, and also to estimate the error of our calculated value relative to the true value.

We define an indicator function $f(x)$ which takes a value of one inside the sphere and value of zero outside of it. Then the volume of the sphere is given by:
\begin{equation}
	V = \int_{|x|\leq 1}f(x) dx
\end{equation}
Where $x \in \mathbb{R}^{10}$ and the integral is over all dimensions. Using the Monte Carlo mean value method, we consider a cubic integration region $[-1,1]^{10}$, which has a volume of $2^{10}$, and use a collection of $N$ randomly-generated points $x_i$ contained in the volume at which to evaluate $f$. The value of our integral (and thus the volume of the hypersphere) is then approximated by:
\begin{equation}
	I = \frac{2^{10}}{N}\sum_{i=1}^N f(x_i)
\end{equation}

The variance of the value of the function $f(x)$ is given by:
\begin{equation}
	\text{var}f = \langle f^2 \rangle - \langle f \rangle^2
\end{equation}
Where $\langle a \rangle$ represents the average value of the variable $a$. The variance of the sum of $f(x)$ over $N$ points is then $N\cdot\text{var}f$, and the standard deviation of the sum is $\sqrt{N\cdot\text{var}f}$. Hence the standard deviation of the volume is:
\begin{equation}
	\sigma = \frac{2^{10}}{N}\sqrt{N\cdot\text{var}f} = \frac{2^{10}\sqrt{\text{var}f}}{\sqrt{N}}
\end{equation}
And we take this standard deviation to be our estimate of the error in the calculated value of the volume.

The values of the volume and associated error output by the program are presented in table \ref{tab:q2}. The analytic value of the volume of a unit sphere in 10 dimensions is, according to Wikipedia, 2.550, and our calculated value is within our calculated error of this, suggesting that we have done the computation correctly.

\begin{table}[H]
	\centering
	\caption{Values of the volume of the hypersphere and error associated with the calculation as output by the program.}
	\label{tab:q2}
	\begin{tabular}{c|c}
		Quantity & Value \\
		\hline
		Volume & 2.567 \\
		Error & 0.051
	\end{tabular}
\end{table}

\section{Importance Sampling}

\end{document}