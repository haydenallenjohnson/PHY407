\documentclass{article}
\usepackage[utf8]{inputenc}
\usepackage{graphicx}
\usepackage{geometry}
\usepackage{amsmath}
\usepackage{amsfonts}
\usepackage{float}
\usepackage{caption}
\usepackage{subcaption}
\usepackage{enumitem}

\geometry{left=25mm, top=25mm, right=25mm, bottom=25mm}

\title{PHY407 Lab 10}
\author{Pierino Zindel (1002429703) and Hayden Johnson (1002103537)}
\date{November 23, 2018}

\begin{document}

\maketitle

\noindent \textbf{Distribution of work:} Question 1 was completed by Pierino. Questions 2 and 3 were completed by Hayden.

\section{Brownian Motion and Diffusion Limited Aggregation}

\section{Volume of a 10-dimensional Hypersphere}

We seek to compute the volume of a 10-dimensional hypersphere, using the Monte Carlo mean value integration method, and also to estimate the error of our calculated value relative to the true value.

We define an indicator function $f(x)$ which takes a value of one inside the sphere and value of zero outside of it. Then the volume of the sphere is given by:
\begin{equation}
	V = \int_{|x|\leq 1}f(x) dx
\end{equation}
Where $x \in \mathbb{R}^{10}$ and the integral is over all dimensions. Using the Monte Carlo mean value method, we consider a cubic integration region $[-1,1]^10$, which has a volume of $2^10$, and use a collection of $N$ randomly-generated points $x_i$ contained in the volume at which to evaluate $f$. The value of our integral (and thus the volume of the hypersphere) is then approximated by:
\begin{equation}
	I = \frac{2^{10}}{N}\sum_{i=1}^N f(x_i)
\end{equation}


\begin{table}[H]
	\centering
	\caption{Values of the volume of the hypersphere and error associated with the calculation as output by the program.}
	\label{tab:q2}
	\begin{tabular}{c|c}
		Quantity & Value \\
		\hline
		Volume & 2.605 \\
		Error & 0.052
	\end{tabular}
\end{table}

\section{Importance Sampling}

\end{document}