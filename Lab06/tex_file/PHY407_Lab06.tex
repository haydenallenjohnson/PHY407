\documentclass{article}
\usepackage[utf8]{inputenc}
\usepackage{graphicx}
\usepackage{geometry}
\usepackage{amsmath}
\usepackage{amsfonts}
\usepackage{float}
\usepackage{caption}
\usepackage{subcaption}
\usepackage{enumitem}

\geometry{left=25mm, top=25mm, right=25mm, bottom=25mm}

\title{PHY407 Lab 5}
\author{Pierino Zindel (1002429703) and Hayden Johnson (1002103537)}
\date{October 12, 2018}

\begin{document}

\maketitle

\noindent \textbf{Distribution of work:}

\section{Modeling space garbage (Newman 8.8)}

For this question we desire to model the trajectory of a ball bearing moving around a rod in space. We assume a negligible cross section of the rod and enough mass to avoid movement. Additionally, the trajectory of the ball bearing is taken to be in a plane perpendicular to the rod with $z=0$.

From part a) of exercise 8.8, the two second order odes that describe the orbit of the ball bearing are given as 

\begin{equation}
	\frac{d^2x}{dt^2} = f_x = -GM\frac{x}{r^2\sqrt{r^2 + L^2/4}}
	\label{eq:1}
\end{equation}
and 
\begin{equation}
	\frac{d^2y}{dt^2} = f_y = -GM\frac{y}{r^2\sqrt{r^2 + L^2/4}}
	\label{eq:2}
\end{equation}

where $r=\sqrt{x^2+y^2}$.

The two second-order odes were then converted to four first-order odes with 
\begin{equation}
	\frac{dv_x}{dt} = f_x 
\end{equation}
\begin{equation}
	\frac{dx}{dt} = v_x
\end{equation}
\begin{equation}
	\frac{dv_y}{dt} = f_y 
\end{equation}
\begin{equation}
	\frac{dy}{dt} = v_y 
\end{equation}

The system of equations was then solved using the 4th-order Runge-Kutta method with the initial conditions $x=1$, $y=0$, $v_x=0$, $v_y=1$; constant values $G=1$, $M=10$, $L=2$, and integrated over the interval $0<t<10$ with $1000$ steps. The resulting orbit trajectory of the ball bearing was then plotted and is shown in figure \ref{fig:q1_orbit}. We see that the resulting orbit is not a simple circular orbit but rather a precessing orbit as suggested in the textbook due to the rod having a non-spherical shape.


\begin{figure}[H]
	\centering
	\includegraphics[width=0.8\textwidth]{../images/q1_orbit.png}
	\caption{Precessing orbit of a ball bearing around a rod, in a plane perpendicular to the rod. Equations of motion of the ball given by equations \ref{eq:1} and \ref{eq:2} with initial conditions of $(x,y,v_x,v_y)=(1,0,0,1)$.}
	\label{fig:q1_orbit}
\end{figure}


\end{document}
