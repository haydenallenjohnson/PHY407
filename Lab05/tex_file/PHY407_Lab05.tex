\documentclass{article}
\usepackage[utf8]{inputenc}
\usepackage{graphicx}
\usepackage{geometry}
\usepackage{amsmath}
\usepackage{amsfonts}
\usepackage{float}
\usepackage{caption}
\usepackage{subcaption}
\usepackage{enumitem}

\geometry{left=25mm, top=25mm, right=25mm, bottom=25mm}

\title{PHY407 Lab 5}
\author{Pierino Zindel (1002429703) and Hayden Johnson (1002103537)}
\date{October 12, 2018}

\begin{document}

\maketitle

\noindent \textbf{Distribution of work:}

\section{More important stuff}

\includegraphics{../images/name.png}

\section{Newman 7.9: Image Deconvolution}
The Fourier transform is very useful for filtering out noise from a given data set. Extending this principle to a set of two dimensional data that corresponds to an image, we can perform a deconvolution of the image and output a sharper image. This question does exactly this by computing the Fourier transform of a blurry image and a point spread function, combining the two through the relation $a_{kl} = \frac{b_{kl}}{f_{kl}}$ where $a_{kl}$ is the Fourier transform of the clear image data, $b_{kl}$ is the Fourier transform of the blurry image data, and $f_{kl}$ is the Fourier transform of the point spread function.

\subsection{Part a)}
Using the blur.txt data file from Newman's Computational Physics site, a greyscale density plot of the data is shown in figure \ref{fig:blurred_image}. The image appears with high resolution but out of focus which allows us to sharpen the image via this process.

\begin{figure}[H]
	\centering
	\includegraphics{../images/blurred_image.png}
	\caption{Density plot of provided data shows a blurred image in greyscale.}
	\label{fig:blurred_image}	
\end{figure}

\subsection{Part b)}
The Gaussian point spread function $f(x,y) = exp(-\frac{x^2+y^2}{2\sigma^2}$, periodic in the interval of the original data, was computed for each point of the corresponding blurred image, starting with $(x=0,y=0)$ in the top-left corner. A brightness graph of the spread function is shown in figure \ref{fig:spread_func}

\begin{figure}[H]
	\centering
	\includegraphics{../images/spread_function.png}
	\caption{Plot of the Gaussian point spread function with $\sigma=25$ used to sharpen the image.}
	\label{fig:spread_func}	
\end{figure}

\subsection{Part c)}
Using the results from part a) and b) we applied a Fourier transform, using the numpy function rfft2, to both the blurred image data set and the spread function data set.
The fft of the blurred image was then divided by the fft of the spread function as described above. To account for the small and possibly zero valued entries in the spread function data set, a cutoff value of $10^{-3}$ was used (wherein the spread function was not applied if the entry values was below the cutoff). Additionally, the textbook notes that the blurred data set should also be divided by the dimensions of the data set, however providing this additional factor resulting in a failure of the deconvolution.
The adjusted fft data was then pass through the inverse, irfft2, function to retrieve a sharpened data set of the original image, which has been graphed in the density plot shown in figure \ref{fig:deconvolved_image}. The resulting image still contains some noise within it however we can now clearly see that the original image was taken of a house with a couple of people walking by in front of it. 

\begin{figure}[H]
	\centering
	\includegraphics{../images/deconvolved_image.png}
	\caption{Density plot of the greyscale image after applying the spread function via Fourier transform.}
	\label{fig:deconvolved_image}	
\end{figure}

\end{document}